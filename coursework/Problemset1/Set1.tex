\documentclass[11pt, english]{article}
\usepackage[english]{babel}
\usepackage[utf8]{inputenc}



\usepackage{geometry}
 \geometry{
 a4paper,
 left=20mm,
 top=30mm,
 right=20mm
 }


\usepackage{listings}
\usepackage{amsmath}
\usepackage{amsfonts}
\usepackage{amssymb}
\usepackage{amsthm} 
\usepackage{mathrsfs}
\usepackage{mathabx}
\usepackage{graphicx}
\usepackage{eurosym}
\usepackage{subfigure}
\usepackage{dsfont}
\usepackage{bbm}

\newcommand{\grafico}[5]{
\begin{figure}
[h!tbp]
\centering
\includegraphics[scale=#2, angle=#3]{#1}
%\captionsetup{width=13cm}
\caption{#4\label{#5}}
\end{figure}
}

\setlength{\parindent}{0pt}

\title{Machine Learning Exercises: Set 1}
\author{Roger Garriga Calleja}
\date{January 30, 2017}

\begin{document}
\maketitle

\textbf{Problem 1: Let $X$ be a real-valued random variable with mean $m$, median $M$, and standard deviation. Prove that}
$$|m-M|\leq\sqrt{2}\sigma.$$

Chebyshev's inequality states that $P(|X-\mathbb{E}X|\geq t)\leq \frac{\sigma^2}{t^2}$

First of all consider $|X-m|$ and apply Chevyshev with $t=\sqrt{2}\sigma$
$$P(|X-m|\geq\sqrt{2}\sigma)\leq\frac{\sigma^2}{2\sigma^2}=\frac{1}{2}.$$
That means that $P(X\notin [m-\sqrt{2}\sigma,m+\sqrt{2}\sigma])\leq\frac{1}{2}$. Taking the complementary we get that $P(X\in [m-\sqrt{2}\sigma,m+\sqrt{2}\sigma])\geq\frac{1}{2}$, which means that the median $M$ will be in the region, so $M\in [m-\sqrt{2}\sigma,m+\sqrt{2}\sigma]$ $\Rightarrow$ $|m-M|\leq \sqrt{2}\sigma$.\\

\textbf{Problem 2: Write a program that compares the performance of the empirical mean and the median-of-means mean estimators. Test them on randomly generated samples drawn from different distributions, including heavy-tailed ones. Try different parameters of the median-of-means estimator and different sample sizes. Compare the estimators according to different measures, such as average deviation from the true mean, as well as worst case deviation (when the random sample is re-drawn many times). You may consider the Pareto family or Student’s $t$-distribution (with different degrees of freedom) for heavy-tailed examples.\\}

We tested the performance of MoM and sample mean estimators using different distributions (Normal, Binomial, Poisson, Gamma, Pareto, T-Student), varying the precision parameter on MoM ($\delta$), the sample size (n) and the parameters of the distributions. In order to test the performance of the estimators we generate a number ($num$) of samples and compute the average deviation and the worst case deviation.\\

On the tables we will use the next notation
\begin{itemize}
	\item Parameters of the distribution in a parenthesis
	\item Sample size: $n$
	\item Precision of MoM: $\delta$
	\item Number of divisions in MoM: $K$
	\item Number of samples: $N$
	\item Expected value: $Exp$
	\item Average sample mean: $Mean$
	\item Average MoM: $MoM$
	\item Worst case deviation: WC (mean) and WC (MoM)
	\item Average deviation: Dev (mean) and Dev (MoM)

\end{itemize}



\textbf{\textit{Light tailed distributions:}}\\

\textit{Normal:}

\begin{center}
	
	\begin{tabular}{|l|l|l|l|l|l||l|l|l|l|l|l|}
		\hline
		($\mu,\sigma^2$) & $n$ & $\delta$ & $K$ & $N$ & $Exp$ & $M$ & $MoM$ & WC $M$ & WC $MoM$ & Dev $M$ & Dev $MoM$ \\
		\hline\hline
		(0,1) & $10^3$ &  $0.05$ & 23 & 100 & 0 & 0.0002 & 0.0009 & 0.0728 & 0.125 & 0.0242 & 0.034\\
		\hline
		(0,1) & $10^3$ &  $0.05$ & 23 & $10^3$ & 0 & -0.0008 & -0.0006 & 0.125 & 0.155 & 0.0256 & 0.031\\
		\hline
		(0,1) & $10^3$ &  $0.3$ & 9 & $10^3$ & 0 & -0.0004 & 0.001 & 0.0971 & 0.157 & 0.0253 & 0.0308\\
		\hline
		(0,1) & $10^3$ &  $0.05$ & 23 & $10^4$ & 0 & 0.0003 & 0.0002 & 0.126 & 0.17 & 0.0251 & 0.031\\
		\hline
		(0,$10^3$) & $10^3$ &  $0.05$ & 23 & $10^4$ & 0 & 0.347 & 0.0036 & 119 & 162
		& 25 & 31.4\\
		\hline
		(0,$10^3$) & $10^3$ &  $0.001$ & 55 & $10^4$ & 0 & 0.0486 & 0.0074 & 124 & 159
		& 25.2 & 31.5\\
		\hline
	\end{tabular}
	
\end{center}


\textit{Poisson:}

\begin{center}
	
	\begin{tabular}{|l|l|l|l|l|l||l|l|l|l|l|l|}
		\hline
		$\lambda$ & $n$ & $\delta$ & $K$ & $N$ & $Exp$ & $M$ & $MoM$ & WC $M$ & WC $MoM$ & Dev $M$ & Dev $MoM$ \\
		\hline\hline
		1 & $10^3$ &  $0.05$ & 23 & 100 & 1 & 1 & 0.999 & 0.087 & 0.0909 & 0.0228 & 0.0307\\
		\hline
		1 & $100$ &  $0.05$ & 23 & $10^3$ & 1 & 1 & 0.966 & 0.35 & 0.4 & 0.0787 & 0.0768\\
		\hline
		15 & $10^3$ &  $0.4$ & 7 & $10^3$ & 15 & 15 & 15 & 0.398 & 0.483 & 0.0987 & 0.12 \\
		\hline
	\end{tabular}
	
\end{center}

\textbf{\textit{Heavy tailed distributions:}}\\

\textit{Pareto:}

\begin{center}
	
	\begin{tabular}{|l|l|l|l|l|l||l|l|l|l|l|l|}
		\hline
		$\alpha$ & $n$ & $\delta$ & $K$ & $N$ & $Exp$ & $M$ & $MoM$ & WC $M$ & WC $MoM$ & Dev $M$ & Dev $MoM$ \\
		\hline\hline
		1.1 & $10^3$ &  $0.05$ & 23 & $10^3$ & 10 & 7.35 & 3.46 & 353 & 7.65 & 6 & 6.54\\
		\hline
		1.1 & $10^4$ &  $0.05$ & 23 & $10^3$ & 10 & 8.07 & 4.7 & 434 & 6.27 & 5.04 & 5.3\\
		\hline
		1.1 & $10^4$ &  $0.001$ & 73 & $10^4$ & 10 & 7.98 & 4.08 & 1.5$\cdot 10^5$ & 6.87 & 5.03 & 5.92\\
		\hline
		1.5 & $10^4$ &  $0.001$ & 73 & $10^3$ & 2 & 2 & 1.73 & 21.4 & 0.503 & 0.166 & 0.278\\
		\hline
		1.5 & $10^3$ &  $0.2$ & 12 & $10^3$ & 2 & 2.06 & 1.7 & 29.4 & 0.81
		& 0.373 & 0.318\\
		\hline
		1.5 & $20$ &  $0.5$ & 5 & $10^3$ & 2 & 1.97 & 1.17 & 179 & 2.84
		& 1.08 & 0.94\\
		\hline
		2 & $20$ &  $0.5$ & 5 & $10^3$ & 1 & 1.03 & 0.744 & 17 & 2.45
		& 0.393 & 0.352\\
		\hline
		2 & $10^3$ &  $0.1$ & 18 & $10^3$ & 1 & 0.996 & 0.923 & 0.742 & 0.298
		& 0.0703 & 0.09\\
		\hline
	\end{tabular}
	
\end{center}

\textit{T-Student:}

\begin{center}
	
	\begin{tabular}{|l|l|l|l|l|l||l|l|l|l|l|l|}
		\hline
		$\nu$ & $n$ & $\delta$ & $K$ & $N$ & $Exp$ & $M$ & $MoM$ & WC $M$ & WC $MoM$ & Dev $M$ & Dev $MoM$ \\
		\hline\hline
		1 & $10^3$ &  $0.05$ & 23 & $10^3$ & 0 & -4.89 & -0.027 & 4310 & 1.27 & 8.05 & 0.274\\
		\hline
		1 & $100$ &  $0.1$ & 18 & $10^3$ & 0 & -0.444 & -0.00356 & 321 & 1.54 & 3.37 & 0.29\\
		\hline
		2 & $100$ &  $0.1$ & 18 & $10^3$ & 0 & -0.0044 & -0.001 & 8.01 & 0.781 & 0.248 & 0.166\\
		\hline
		2 & $10^3$ &  $0.01$ & 36 & $10^3$ & 0 & 0.0045 & 0.00257 & 0.826 & 0.361 & 0.0879 & 0.0701\\
		\hline
		5 & $100$ &  $0.1$ & 18 & $10^3$ & 0 & -0.00852 & -0.00511 & 0.446 & 0.509
		& 0.104 & 0.122\\
		\hline
	\end{tabular}
	
\end{center}


As can be seen in the tables, in average the sample mean is closer than MoM to the expected value, but not much closer. However, if we look at the worst case deviation, there are cases in which the sample mean has a huge deviation but the MoM remains fairly close to the expected value. \\
For the light-tailed distributions it would be better to use the sample mean as estimator. Whereas for the heavy-tailed ones it may be dangerous to use the sample mean because there are time when the deviation from the expected value is very large. If we are sure that the underlying distribution of our data is light-tailed we can use the sample mean, otherwise using it may lead to big mistakes, so MoM is much more conservative in this sense. \\ 



\textbf{Problem 3: Let $X_1,\dots,X_n$ be i.i.d. \textit{non-negative} random variables with mean $\mathbb{E}X_1=m$ and second moment $\mathbb{E}X_1^2=a^2$. Use the Chernoff bound to prove that, for all $t\in(0,m)$,}
$$P\left\{\frac{1}{n}\sum_{i=1}^n X_i<m-t\right\}\leq e^{-\frac{n t^2}{2 a^2}}.$$
\textbf{\textit{Hint:} use the fact that for $x>0$, $e^{-x}\leq 1-x+\frac{x^2}{2}$.\\}

Working a bit the equation we get $$P\left(\frac{1}{n}\sum_{i=1}^nX_i<m-t\right)=P\left(t<m-\frac{1}{n}\sum_{i=1}^nX_i\right)=P\left(m-\frac{1}{n}\sum_{i=1}^nX_i>t\right)=P\left(\sum_{i=1}^n\mathbb{E}X-\sum_{i=1}^nX_i>nt\right).$$
Now we can apply Chernoff bound taking into account that $X_i$ are independent, so
$$P\left(\sum_{i=1}^n\mathbb{E}X-\sum_{i=1}^nX_i>nt\right)\leq \frac{\mathbb{E}\prod\limits_{i=1}^ne^{\lambda(\mathbb{E}X_i-X_i)}}{e^{\lambda t n}}.$$
Then, as we have $\mathbb{E}X_i=m$ $\forall i$, we get
$$P\left(\sum_{i=1}^n\mathbb{E}X-\sum_{i=1}^nX_i>nt\right)\leq \frac{\mathbb{E}\prod\limits_{i=1}^ne^{\lambda(\mathbb{E}X_i-X_i)}}{e^{\lambda t n}}=\frac{1}{e^{n\lambda t}}\prod\limits_{i=0}^n\mathbb{E}e^{\lambda(m-X_i)}=\frac{e^{n\lambda m}}{e^{n\lambda t}}\prod\limits_{i=0}^n\mathbb{E}e^{-\lambda X_i}.$$
As $e^{-x}\leq 1-x+\frac{x^2}{2}$ for $x>0$ and our $X_i$ are non-negative, we get
$$\frac{e^{n\lambda m}}{e^{n\lambda t}}\prod\limits_{i=0}^n\mathbb{E}e^{-\lambda X_i}\leq e^{n\lambda (m-t)}\prod\limits_{i=1}^n(1-\lambda \mathbb{E}X_i+\frac{\lambda^2}{2}\mathbb{E}X_i^2).$$
Now, since $1+x\leq e^x$ $\forall x\in \Re$ (on $x=0$ both are equal and $e^x$ derivative is greater or equal to 1 for $x\geq0$ and less than 1 for $x<0$), we get
$$e^{n\lambda (m-t)}\prod\limits_{i=1}^n(1-\lambda \mathbb{E}X_i+\frac{\lambda^2}{2}\mathbb{E}X_i^2)\leq e^{n\lambda(m-t)}\prod\limits_{i=1}^n e^{-\lambda m+\frac{\lambda^2}{2}a^2}=e^{n\lambda(m-t)}e^{n(-\lambda m+\frac{\lambda^2}{2}a^2)}=e^{-n\lambda t+n\frac{\lambda^2}{2}a^2}.$$
We minimize the function on $\lambda$ to get the bound. $0=(-n\lambda t+n\frac{\lambda^2}{2}a^2)'=-nt+n\lambda a^2$ $\Leftrightarrow \lambda=\frac{t}{a^2}$. That implies 
 
$$e^{-n\lambda t+n\frac{\lambda^2}{2}a^2}\leq e^{-\frac{nt^2}{2a^2}}.$$

So, $P\left(\frac{1}{n}\sum_{i=1}^nX_i<m-t\right)\leq e^{-\frac{nt^2}{2a^2}}$. Q.E.D.\\

\textbf{Problem 4: Write a program that projects the n standard basis vectors in $\Re^n$ to a random 2-dimensional subspace. (You may do this simply by using a $2\times n$ matrix whose entries are i.i.d. normals.) Center the point set appropriately and re-scale such that the empirical variance of the (say) first component equals 1. Plot the obtained point set. Now generate n independent standard normal vectors on the plane and compare the two plots. Do this for a wide range of values of $n$. What do you see?\\
Repeat the same exercise but now projecting the $2n$ vertices of the hypercube $\{1,1\}^n$ instead
of the standard basis vectors. (Naturally, you can only do this for small values of $n$, say up to $n\simeq 13$.) What do you see now?\\}

The projected standard basis vectors of $\Re^n$ to a 2-dimensional subspace are simply random vectors following a normal distribution. Intuitively we could prognosticate that, because we multiply a random matrix whose entries are standard normal with parameters $\mu=0$, $\sigma^2=\frac{1}{d}$ with an identity matrix. After centering and rescaling we make force the entries to be normal with parameters $\mu=0$, $\sigma^2=1$.


\grafico{Graphics/proj_basis.png}{0.4}{0}{Projected basis (blue) and standard normal vectors (red).}{basis}
\newpage
The projected hypercubes to a 2-dimensional subspace preserve the structure they had before being projected. The projection is like cutting the D-space on a random plane and project the hypercube on it. The result is that the points that the points keep structure on the pairwise distance, in the sense that if the distance between a pair of points is the same as the distance between another pair of points, in the projections they will have also the same distance. The ones that are the closest remain the closest. 


\grafico{Graphics/proj_hype.png}{0.8}{0}{Projected hypercube (blue).}{hypercube}

\newpage

\section{Appendix}

\textbf{Ex 2:}

\begin{lstlisting}[language=Python]

import sklearn
import math as mat
import numpy as np
import itertools as itt
import operator as op
import pandas as pd
import matplotlib.pyplot as plt
from numpy import random as rand
import functools as ft

#Given a sample (data) and a precision (delta). Computes de MoM estimator with 
#K=[8*log(1/delta)] divisions.
def MoM(data,delta=0.1):
	K=int(8*mat.log(1/delta)) #compute num of divisions.
	rand.shuffle(data) #shuffle the data.
	sp_data=np.array_split(data,K) #split the data into K sets. 
	means=list(map(np.mean,sp_data)) #computes the mean in each set.
	return np.median(means) #computes the median of the means.


#distr: Normal(mu,var), Binomial(n,p), Poisson(lbd), Gamma(k,theta), 
#Pareto(a), T-Student(nu), Weibull, Lognorm
#parm1 and parm2: parameters of the distribution
#n: number of random numbers in each sample
#draws: number of times it is going to be performed
#delta: precision
def perf_MoM(distr,parm1,parm2,n,draws,delta):
	#replicate the parameters a number of times indicated by the draws.    
	p1=[parm1]*draws
	p2=[parm2]*draws
	num=[n]*draws
	prec=delta
	if distr=="Normal": #Generates 'draws' iid Normal samples of 'num' elements
		data=list(map(rand.normal,p1,p2,num)) 
		exp_value=[parm1]*draws
	elif distr=="Binomial": #Generates 'draws' iid Binomial samples of 'num' elements
		data=list(map(rand.binomial,p1,p2,num))
		exp_value=[parm1*parm2]*draws
	elif distr=="Poisson": #Generates 'draws' iid Poisson samples of 'num' elements
		data=list(map(rand.poisson,p1,num))
		exp_value=[parm1]*draws
	elif distr=="Gamma": #Generates 'draws' iid Gamma samples of 'num' elements
		data=list(map(rand.gamma,p1,p2,num))
		exp_value=[parm1*parm2]*draws
	elif distr=="Pareto": #Generates 'draws' iid Pareto samples of 'num' elements
		data=list(map(rand.pareto,p1,num))
		exp_value=[parm1/(parm1-1)-1]*draws
	elif distr=="T-Student": #Generates 'draws' iid T-student samples of 'num' elements
		data=list(map(rand.standard_t,p1,num))
		exp_value=[0]*draws
	elif distr=="Weibull": #Generates 'draws' iid Weibull samples of 'num' elements
		data=list(map(op.mul,list(map(rand.weibull,p1,num)),p2))
		exp_value=[parm2*(mat.gamma(1+1/parm1))]*draws
	elif distr=="Lognorm": #Generates 'draws' iid Lognorm samples of 'num' elements
		data=list(map(rand.lognormal,p1,p2,num))
		exp_value=[mat.exp(parm1+parm2/2)]*draws
	MoM_est=list(map(ft.partial(MoM,delta=prec),data)) #Compute the 'draws' estimators using MoM
	mean_est=list(map(np.mean,data)) #Compute the 'draws' estimators using sample mean
	dif_mean=list(map(abs,map(op.sub,exp_value,mean_est))) #Compute the error of each sample mean
	dif_MoM=list(map(abs,map(op.sub,exp_value,MoM_est))) #Compute the error of each MoM
	#eval performance using worst case    
	wc_mean=max(dif_mean) 
	wc_MoM=max(dif_MoM)
	#eval performance using best case
	bc_mean=min(dif_mean)
	bc_MoM=min(dif_MoM)    
	#eval performance using average of error
	ave_mean=sum(dif_mean)/len(dif_mean)
	ave_MoM=sum(dif_MoM)/len(dif_MoM)
	#compute the mean of the estimators over the 'draws' samples.
	ave_mean_MoM=np.mean(MoM_est)
	ave_mean_mean=np.mean(mean_est)
	#plt.hist(MoM_est,color="blue")
	#plt.hist(mean_est,color="green")
	compare=pd.DataFrame([['','Exp value','Worst case dev','Best case dev',
	'Average dev','ave_mean'],
	['Mean',exp_value[0],wc_mean,bc_mean,ave_mean,ave_mean_mean],
	['MoM',exp_value[0],wc_MoM,bc_MoM,ave_MoM,ave_mean_MoM]])
	return compare

\end{lstlisting}

\textbf{Ex 4:}

\begin{lstlisting}
import numpy as np
import itertools as itt
import matplotlib.pyplot as plt
from numpy import random as rand
	
def proj_basis(d,D): #Projects the basis of a D-dim space to a d-dim space
	W=rand.normal(0,1/d,(d,D)) #Generate a random matrix to project D-dim vectors to d-dim space 
	basis=np.identity(D) #Generate the basis of a D-dim space
	proj_vect=np.dot(W,basis) #Project the basis
	proj_vect[0]=proj_vect[0]-np.mean(proj_vect[0]) #center first component
	proj_vect[1]=proj_vect[1]-np.mean(proj_vect[1]) #center second component
	std_dev=np.sqrt(np.var(proj_vect[0,])) #compute the std dev of the first component 
	proj_vect=proj_vect/std_dev #rescale by first component
	return proj_vect
	
d=2
i=0
rng=[10,50,100,150,200,500,1000,10000]
for D in rng: #Plot the proj basis for a rng of dimensions D into a d-dim space
	i=i+1    
	proj_vect=proj_basis(d,D)
	rnd_vect_plane=rand.normal(0,1,(2,D)) #generate random normals
	plt.subplot(4,2,i) #more than one plot
	plt.scatter(proj_vect[0],proj_vect[1]) 
	plt.scatter(rnd_vect_plane[0],rnd_vect_plane[1],color="red")
	plt.title("N=%d"%D) #change title
	
#hypercube
	
def proj_hypercube(d,D):	
	vmax=[1]*D
	vmin=[-1]*D
	
	hypercube=np.transpose(np.asarray(list(itt.product(*zip(vmin,vmax))))) #generates the vertices
	W=rand.normal(0,1/d,(d,D)) #Generates the projection matrix
	proj_hyp_cube=np.dot(W,hypercube) #Projects
	proj_hyp_cube[0]=proj_hyp_cube[0]-np.mean(proj_hyp_cube[0])
	proj_hyp_cube[1]=proj_hyp_cube[1]-np.mean(proj_hyp_cube[1])
	std_dev=np.sqrt(np.var(proj_hyp_cube[1,]))
	proj_hyp_cube=proj_hyp_cube/std_dev
	
	return proj_hyp_cube
	
d=2
rng=[2,3,4,5,6,10]
i=0
for D in rng: #projects the hypercubes from different dimensions to a 2-dim subspace
	i=i+1
	proj_hyp_cube=proj_hypercube(d,D)	
	plt.subplot(3,2,i) #more than one plot
	plt.scatter(proj_hyp_cube[0],proj_hyp_cube[1])
	plt.title("D=%d"%D) #change title
	
	
\end{lstlisting}


\end{document}